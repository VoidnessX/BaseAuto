\documentclass[twoside]{article}

\usepackage{lipsum} % Package to generate dummy text throughout this template

\usepackage[sc]{mathpazo} % Use the Palatino font
\usepackage[T1]{fontenc} % Use 8-bit encoding that has 256 glyphs
\linespread{1.05} % Line spacing - Palatino needs more space between lines
\usepackage{microtype} % Slightly tweak font spacing for aesthetics

\usepackage[hmarginratio=1:1,top=32mm,columnsep=20pt]{geometry} % Document margins
\usepackage{multicol} % Used for the two-column layout of the document
\usepackage[hang, small,labelfont=bf,up,textfont=it,up]{caption} % Custom captions under/above floats in tables or figures
\usepackage{booktabs} % Horizontal rules in tables
\usepackage{float} % Required for tables and figures in the multi-column environment - they need to be placed in specific locations with the [H] (e.g. \begin{table}[H])
\usepackage{hyperref} % For hyperlinks in the PDF

\usepackage{lettrine} % The lettrine is the first enlarged letter at the beginning of the text
\usepackage{paralist} % Used for the compactitem environment which makes bullet points with less space between them

\usepackage{multirow}
\usepackage{braket}
\usepackage{array}
\usepackage{calc}
\usepackage{graphicx}
\usepackage{listings}
\usepackage{color}
\usepackage[table,xcdraw]{xcolor}
\usepackage{adjustbox}
\usepackage{datetime}


\definecolor{dkgreen}{rgb}{0,0.6,0}
\definecolor{gray}{rgb}{0.5,0.5,0.5}
\definecolor{mauve}{rgb}{0.58,0,0.82}



\hypersetup{%
    pdfborder = {0 0 0}
}



\usepackage{abstract} % Allows abstract customization
\renewcommand{\abstractnamefont}{\normalfont\bfseries} % Set the "Abstract" text to bold
\renewcommand{\abstracttextfont}{\normalfont\small\itshape} % Set the abstract itself to small italic text

\usepackage{titlesec} % Allows customization of titles
\renewcommand\thesection{\Roman{section}} % Roman numerals for the sections
\renewcommand\thesubsection{\Roman{subsection}} % Roman numerals for subsections
\titleformat{\section}[block]{\large\scshape\centering}{\thesection.}{1em}{} % Change the look of the section titles
\titleformat{\subsection}[block]{\large}{\thesubsection.}{1em}{} % Change the look of the section titles

\usepackage{kotex}
\usepackage{fancyhdr} % Headers and footers
\pagestyle{fancy} % All pages have headers and footers
\fancyhead{} % Blank out the default header
\fancyfoot{} % Blank out the default footer
\fancyhead[C]{ 2015 Summer Project} % Custom header text
\fancyfoot[RO,LE]{\thepage} % Custom footer text

%----------------------------------------------------------------------------------------
%	TITLE SECTION
%----------------------------------------------------------------------------------------

\title{\vspace{-15mm}\fontsize{24pt}{10pt}\selectfont\textbf{Automatic Generation of Baseball Articles  }} % Article title

\author{
\large
\textsc{김동현, 김용현, 신승우, 이경수}\\[2mm]
%\textsc{Seungwoo Schin}\\[2mm]
\normalsize Team Noname \\ % Your institution
\vspace{-5mm}
}
\date{\today}

%----------------------------------------------------------------------------------------

\begin{document}

\maketitle % Insert title

\thispagestyle{fancy} % All pages have headers and footers

%----------------------------------------------------------------------------------------
%	ARTICLE CONTENTS
%----------------------------------------------------------------------------------------




\section{Main Objective}

야구 경기 기록에서 야구 기사를 자동으로 작성하는 코드를 짜는 것. 
야구 경기 기록은 kbo 홈페이지\footnote{http://www.koreabaseball.com/Schedule/GameList/General.aspx}와 네이버 문자중 계 에서 얻어온다.  

\section{General Structure}

\subsection{Pipeline of Program}

\begin{enumerate}
\item 데이터 수집 

웹 크롤링을 통해서 기존 야구 경기의 기록과 관련 기사를 수집한다. 

\item 학습 

경기의 기록을 분석해서 어떤 부분이 기사에 주로 나오는지를 학습한다. 

\item 기사 작성 

기사에 나올 부분을 자연어처리를 통해서 기사로 만들어낸다. 

\end{enumerate}

\subsection{Program Structure}


\subsubsection{경기 구조}

야구 경기는 1구마다, 혹은 교체 등의 예외적인 상황마다 state가 변한다. 이 state가 변하는 과정을 event라고 한다. 따라서 야구 경기는 state들이 변화하는 과정이다. 맨 처음 state는 선발 라인업을 반영해야 하고, 그 이후로는 일어난 event에 따라서 state가 바뀌게 된다. 이 과정을 track함으로써 야구 경기에서 일어난 일을 재현할 수 있다.

State는 다음의 정보를 담고 있다. 
\begin{compactitem}
\item Offense/Defencse Team : 공/수 팀의 포지션별 선수 목록
\item ball, out, strike : 각각 볼, 아웃, 스트라이크 카운트
\item turn, is\_home : 회차와 초/말. 
\end{compactitem}

Event의 종류는 아래와 같이 분류된다. 
% Please add the following required packages to your document preamble:
% \usepackage{multirow}
\begin{table}[]
\centering
\caption{Event class}
\label{my-label}
\begin{tabular}{|l|l|l|l|l|}
\hline
\multirow{11}{*}{hitted}    & \multirow{10}{*}{fair}  & \multirow{5}{*}{hit} & single         & 1루타 \\ \cline{4-5} 
                            &                         &                      & double         & 2루타   \\ \cline{4-5} 
                            &                         &                      & triple         & 3루타   \\ \cline{4-5} 
                            &                         &                      & homerun        & 홈런    \\ \cline{4-5} 
                            &                         &                      & error          & 실책    \\ \cline{3-5} 
                            &                         & \multirow{5}{*}{out} & singleout      & 아웃    \\ \cline{4-5} 
                            &                         &                      & doubleplay     & 병살    \\ \cline{4-5} 
                            &                         &                      & tripleplay     & 삼중살    \\ \cline{4-5} 
                            &                         &                      & sacrifice fly  & 희생 플라이    \\ \cline{4-5} 
                            &                         &                      & sacrifice bunt & 희생 번트    \\ \cline{2-5} 
                            & \multicolumn{3}{l|}{foul}                                       &  파울   \\ \hline
\multirow{7}{*}{Non-hitted} & \multirow{3}{*}{strike} & \multicolumn{2}{l|}{strike}           &  스트라이크   \\ \cline{3-5} 
                            &                         & \multicolumn{2}{l|}{swing}            &   헛스윙  \\ \cline{3-5} 
                            &                         & \multicolumn{2}{l|}{strike out}       &  스트라이크 아웃  \\ \cline{2-5} 
                            & \multirow{4}{*}{ball}   & \multicolumn{2}{l|}{ball}             &  볼   \\ \cline{3-5} 
                            &                         & \multicolumn{2}{l|}{wildpitch}        &  폭투   \\ \cline{3-5} 
                            &                         & \multicolumn{2}{l|}{hit by pitch}     &  데드볼   \\ \cline{3-5} 
                            &                         & \multicolumn{2}{l|}{base on balls}    &  사구   \\ \hline
\multirow{3}{*}{steal}      & \multicolumn{3}{l|}{safe}                                       &  스틸 성공  \\ \cline{2-5} 
							& \multicolumn{3}{l|}{double steal}								  &  더블스틸 \\ \cline{2-5}
                            & \multicolumn{3}{l|}{out}                                        &  도루사    \\ \hline
\multicolumn{4}{|l|}{balk}                                                                    &  보크   \\ \hline
\multirow{5}{*}{change}     & \multicolumn{3}{l|}{PH}                                         &  대타   \\ \cline{2-5} 
                            & \multicolumn{3}{l|}{PR}                                         &  대주자   \\ \cline{2-5} 
                            & \multicolumn{3}{l|}{Pitcher Change}                             &  투수 교체   \\ \cline{2-5} 
                            & \multicolumn{3}{l|}{Fielder Change}                             &  야수 교체   \\ \cline{2-5} 
                            & \multicolumn{3}{l|}{Penalty}                                    &  퇴장   \\ \hline 
\multicolumn{4}{|l|}{Team Change}                                                             &  공수 교대   \\ \hline
\multicolumn{4}{|l|}{Bench change}                                                            &  ?   \\ \hline
\multicolumn{4}{|l|}{Bench Clear}                                                             &  벤치 클리어   \\ \hline
\multicolumn{4}{|l|}{hit interference}                                                        &  타격 방해   \\ \hline
\end{tabular}
\end{table}

event가 일어날 시에 그 event에 의힌 선수들의 stat 변화량을 계산해서 기록한다. stat들은 크게 수비와 공격 stat으로 나눠서 판단한다. 
공격 stat은 wpa를 우선 사용하며, wpa이외의 스탯은 일단 사용하지 않는다. 
수비 스탯의 경우는 subgame의 단위로 게임을 나누어서, 각 subgame을 설명하는 지표를 아래와 같이 정한다. \footnote{스탯 정확히 정해지면 수정 요}


\subsubsection{야구 기사}

야구 기사는 경기에서 일어난 event들 중 특정 event를 기자가 취사선택하여 배열 후 서술한 것이다. 따라서 이 과정을 기계가 대신하게 하는 것으로 기사 작성을 할 수 있다. 이를 위해서는 기존 기사에서 어떤 event를 어떤 순서로 배열하여 작성한지를 알 수 있어야 한다. \footnote{이 부분에 대한 설계는 아직 정확히 진행하지 않았음. 추가 요망. } 


\section{Details}

작업 세부사항. 

\subsection{Development Environment}
 
\begin{compactitem}
\item 언어 : python3, python2
\item 필요 모듈 : BeautifulSoup4, KoNLPy(예정), Numpy/Scipy/Matplotlib(예정)
\item remote server : 157.7.108.38
\item 소스 버젼관리 : 미정
\item 소스 공유 툴 : 미정\footnote{아마 github}
\item 자료 공유 : slack, 카카오톡
\end{compactitem}

\subsection{Build Environment}

\begin{compactitem}
\item 언어 : python2
\item 필요 모듈 : BeautifulSoup4, Selenium 
\item 기타 설치 요 : ChromeDriver
\end{compactitem}
\subsection{Coding Convention} 
기본적으로 PEP에 나와있는 사항을들 따르되, 아래 사항은 필수적으로 지킬 것. 

\begin{compactitem}

\item Naming

\begin{compactitem}
\item 클래스 이름 : 첫 문자 대문자. 이후 단어 단위로 대문자. ex) ExampleClass
\item 함수 이름 : 첫 문자 소문자. 이후 단어 단위로 대문자.\footnote{Camel Notation} ex) exampleFunc
\item 변수 이름 : 모든 문자 소문자. 단어 단위로 \_. ex)example\_var
\item 전역 변수 : 모든 문자 대문자. 단어 단위로 \_. ex)GLOBAL\_VAR
\end{compactitem}

\item Comment

모든 주석은 영어로, 아래의 형식으로 달면 됨. 
\begin{compactitem}
\item 클래스 

\begin{lstlisting}
'''
class ExampleClass : class explanation
	att1 : att1 explanation
	att2 : att2 explanation 
...

'''
class ExampleClass():
	def __init__(self, att1, att2. ...):
		....
\end{lstlisting}

\item 함수 


\begin{lstlisting}
'''
exampleFunc : type(var1), type(var2), ... -> type(res1), type(res2), ...
	function explanation 
	var1 : var1 explanation
	var2 : var2 explanation 
	...
	res1 : res1 explanation 
	res2 : res2 explanation 
	...
'''
def exampleFunc(var1, var2, ...):
	...
	
\end{lstlisting}

필요한 경우, 함수 내부에 local variable을 설명하는 주석을 달아주면 좋다. 

\item 전역변수 

전역변수 정의 위에 설명 comment 한 줄. 
\begin{lstlisting}
# GLOBAL_VARIABLE : explanation 
GLOBAL_VARIABLE = ...
\end{lstlisting}

\end{compactitem}
\end{compactitem}


\subsection{Implementation}

별지 첨부. 

\section{기타}

\subsection{Team Members}

모든 사람 명단은 가나다순으로 쓰여졌음. 

\begin{compactitem}
\item 김동현

서버세팅, 코딩 담당. 
\item 김용현

코딩, 야구 관련 지식 담당. 
\item 신승우 

문서화, 코딩 담당. 
\item 이경수 

문서화, 야구 관련 지식 담당. 
\end{compactitem}

\subsection{Project History}


\paragraph{\textbf{2015.07.04 11:00-4:00}}

\begin{compactitem}
\item 참가자 : 김동현, 신승우, 이경수
\item 한 일 : 전반적인 아키텍쳐 논의. 개발환경 협의. 서버 세팅.
\item 합의사항 : 모임장소는 신촌 cafe.blog로 정함. 모임 시간은 매주 토요일 12:00까지 오기로. 
\item 기타 : 없음. 
\end{compactitem}

\paragraph{\textbf{2015.07.11 11:00-4:00}}
\begin{compactitem}
\item 참가자 : 김동현, 신승우, 이경수
\item 한 일 : 할 일 분배, pseudo-code implement하기. 
\item 합의사항 : 스케쥴 정리 - 25일까지. 다음주 미팅 없음.  
\item 각자 할 일 (due 25일)
\begin{compactitem}
\item 김동현 : 크롤러 짜보기 
\item 김용현 : 파서 설계해서 pseudocode로 가져올것 
\item 신승우 : 파서 설계/evalState 함수 짜오기/eventClass 짜오기
\item 이경수 : 트레이드 정보 포함한 player list 만들기. 각자에게 배분해줄 것. 
\end{compactitem}
\item 기타 :  없음. 
\end{compactitem}


\paragraph{\textbf{2015.07.26 10:00-6:00}}
\begin{compactitem}
\item 참가자 : 김동현, 신승우, 이경수, 김용현
\item 한 일 : crawler 작성, 스탯벡터 만드는 전처리 
\item 합의사항 : Publish 방법 합의 - 후속결정사항 기록 필. 
\item 각자 할 일 (due 1일)
\begin{compactitem}
\item 김동현 : (야구 기사 -> eventClass) Parser 설계 / 크롤러 디버깅
\item 김용현 : (문자중계 -> eventClass) Parser 설계해서 pseudocode로 가져올것 / 수비스탯 합의해서 가져오기  
\item 신승우 : (야구 기사 -> eventClass) Parser 설계/event handler 마무리 / 크롤러 디버깅 / 소스트리 
\item 이경수 : 수비스탯 합의해서 가져오기 
\end{compactitem}
\item 기타 :  없음. 
\end{compactitem}

\paragraph{\textbf{2015.07.27} - publish 후속관련사항}
\begin{compactitem}
\item 참가자 : 외부회의 
\item 합의사항 
\begin{compactitem} 
\item 일정 관련 : 2주 안에 1차적인 학습, 프로토타입 만들어보기. 리그 끝나기 전에 간단한 기사라도 만들어볼 것. 
\item 저작권, 수익 : 공동분배. 비율은 차후 합의. 초창기 수익모델은 정해지지 않았음. 
\item 기타 : 2주 후 회의. 
\end{compactitem}
\end{compactitem}


\end{document}