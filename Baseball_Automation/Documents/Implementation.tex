\documentclass[twoside]{article}

\usepackage{lipsum} % Package to generate dummy text throughout this template

\usepackage[sc]{mathpazo} % Use the Palatino font
\usepackage[T1]{fontenc} % Use 8-bit encoding that has 256 glyphs
\linespread{1.05} % Line spacing - Palatino needs more space between lines
\usepackage{microtype} % Slightly tweak font spacing for aesthetics

\usepackage[hmarginratio=1:1,top=32mm,columnsep=20pt]{geometry} % Document margins
\usepackage{multicol} % Used for the two-column layout of the document
\usepackage[hang, small,labelfont=bf,up,textfont=it,up]{caption} % Custom captions under/above floats in tables or figures
\usepackage{booktabs} % Horizontal rules in tables
\usepackage{float} % Required for tables and figures in the multi-column environment - they need to be placed in specific locations with the [H] (e.g. \begin{table}[H])
\usepackage{hyperref} % For hyperlinks in the PDF

\usepackage{lettrine} % The lettrine is the first enlarged letter at the beginning of the text
\usepackage{paralist} % Used for the compactitem environment which makes bullet points with less space between them

\usepackage{braket}
\usepackage{array}
\usepackage{calc}
\usepackage{graphicx}
\usepackage{listings}
\usepackage{color}
\usepackage[table,xcdraw]{xcolor}
\usepackage{adjustbox}
\usepackage{kotex}


\definecolor{dkgreen}{rgb}{0,0.6,0}
\definecolor{gray}{rgb}{0.5,0.5,0.5}
\definecolor{mauve}{rgb}{0.58,0,0.82}



\hypersetup{%
    pdfborder = {0 0 0}
}



\usepackage{abstract} % Allows abstract customization
\renewcommand{\abstractnamefont}{\normalfont\bfseries} % Set the "Abstract" text to bold
\renewcommand{\abstracttextfont}{\normalfont\small\itshape} % Set the abstract itself to small italic text

\usepackage{titlesec} % Allows customization of titles
%\renewcommand\thesection{\Roman{section}} % Roman numerals for the sections
\renewcommand\thesubsection{\Roman{subsection}} % Roman numerals for subsections
\titleformat{\section}[block]{\large\scshape\centering}{\thesection.}{1em}{} % Change the look of the section titles
\titleformat{\subsection}[block]{\large}{\thesubsection.}{1em}{} % Change the look of the section titles
\setcounter{section}{-1}

\usepackage{fancyhdr} % Headers and footers
\pagestyle{fancy} % All pages have headers and footers
\fancyhead{} % Blank out the default header
\fancyfoot{} % Blank out the default footer
\fancyhead[C]{ BA101 Baseball Article Writing} % Custom header text
\fancyfoot[RO,LE]{\thepage} % Custom footer text

%----------------------------------------------------------------------------------------
%	TITLE SECTION
%----------------------------------------------------------------------------------------

\begin{document}
\begin{titlepage}

\newcommand{\HRule}{\rule{\linewidth}{0.5mm}} % Defines a new command for the horizontal lines, change thickness here

\center % Center everything on the page
 
%----------------------------------------------------------------------------------------
%	HEADING SECTIONS
%----------------------------------------------------------------------------------------

\vspace*{3cm}
\textsc{\Large BA101}\\[0.5cm] % Major heading such as course name
\textsc{\large Baseball Article Writing}\\[0.5cm] % Minor heading such as course title

%----------------------------------------------------------------------------------------
%	TITLE SECTION
%----------------------------------------------------------------------------------------

\HRule \\[0.4cm]
{ \huge \bfseries Baseball Automation - Details}\\[0.4cm] % Title of your document
\HRule \\[1.5cm]
 
%----------------------------------------------------------------------------------------
%	AUTHOR SECTION
%----------------------------------------------------------------------------------------

\begin{minipage}{0.4\textwidth}
\begin{flushleft} \large
\emph{Author:}\\
신승우, 김동현 \\ 
김용현, 이경수 
\end{flushleft}
\end{minipage}
~
\begin{minipage}{0.4\textwidth}
\begin{flushright} \large
\emph{Typeset by:} \\
신승우
\end{flushright}
\end{minipage}\\[4cm]

% If you don't want a supervisor, uncomment the two lines below and remove the section above
%\Large \emph{Author:}\\
%John \textsc{Smith}\\[3cm] % Your name

\textsc{Team NoName}\\[1.5cm] % Name of your university/college

%----------------------------------------------------------------------------------------
%	DATE SECTION
%----------------------------------------------------------------------------------------

{\large \today}\\[3cm] % Date, change the \today to a set date if you want to be precise


%----------------------------------------------------------------------------------------
%	LOGO SECTION
%----------------------------------------------------------------------------------------

%\includegraphics{Logo}\\[1cm] % Include a department/university logo - this will require the graphicx package
 
%----------------------------------------------------------------------------------------

%\vfill % Fill the rest of the page with whitespace

\end{titlepage}

% Table of contents 

\tableofcontents
\newpage



\lstset{frame=tb,
  language=Python,
  aboveskip=3mm,
  belowskip=3mm,
  showstringspaces=false,
  columns=flexible,
  basicstyle={\small\ttfamily},
  numbers=none,
  numberstyle=\tiny\color{gray},
  keywordstyle=\color{blue},
  commentstyle=\color{dkgreen},
  stringstyle=\color{mauve},
  breaklines=true,
  breakatwhitespace=true
  tabsize=3
}
\section{Introduction}

본 서류는 야구 자동작성 프로젝트에 쓰이는 소스 코드에 대한 내용을 담고 있다. 프로젝트의 모든 내용은 root 디렉토리 안에 있으며, 디렉토리는 다음과 같은 트리구조로 되어 있다. 

\begin{compactitem}
\item backup : 백업 파일. 소스 파일을 백업한다. 
\item documents : 관련 서류를 저장하는 폴더. 
\item etc : 기타 연습용 코드 등을 저장. 프로젝트가 끝난 후 지울 것. 
\item raw : 긁어온 raw data들을 담고 있다. 
\item result : 긁어온 자료를 프로세싱한 결과를 담고 있다. 
\item src : 소스 파일들을 저장하는 폴더. 
\end{compactitem}

소스 코드는 아래의 5가지로 나뉘어진다. \footnote{필요한 그림들을 그려서 넣을 필요가 있음.}

\begin{compactitem}
\item 크롤러 : 각종 사이트에서 필요한 정보를 긁어온다. 
\item 야구 시뮬레이터 : 야구 경기를 프로그램 내에서 시뮬레이트하여 이벤트를 추출한다. 
\item 자연어처리 부분 : 기사에서 필요한 정보를 추출하거나, 만들어낸 이벤트를 이용해서 기사를 만들어낸다.
\item 기계학습 : 기사에 쓸 이벤트를 추출하는 classfier를 만든다.
\item 기타 : 여러 정보를 처리하거나, 실행 자동화를 위한 부분을 담고 있다.  
\end{compactitem}

\section{Source Code Details}

\subsection{Crawler}

크롤러는 크게 2개가 있다. 야구 기사들을 긁어와서 저장하는 crawlLeagueInfo.py와 문자중계를 긁어오는 crawlTextRelay.py이다.\footnote{KBO에서 구질, 구속을 긁어오는 것 역시 필요하다. 아직 구현되지 않았으나 구현 필요.} 

\subsubsection{Article Crawler}

야구 기사 읽어오기. 

\subsubsection{Text Broadcasting Crawler}

문자중계 읽어오기.

\subsection{Simulator}

야구 시뮬레이션은 event 단위로 이루어진다. 이는 곧

\subsection{NLP handling}

기사를 읽어오는 것과 기사를 쓰는 것으로 나누어진다. 

\subsection{Etc}  

몇 가지 기타 사항들. 



\end{document}








